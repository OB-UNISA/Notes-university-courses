\section{Definizioni}
\begin{itemize}
    \item L'alfabeto $\Sigma$ è un insieme di simboli per creare stringhe.
    \item La stringa è una \textbf{sequenza} di simboli presi da un alfabeto $\Sigma$.
    \item $\lambda$ è la \textbf{stringa vuota} che non contiene simboli. $\lambda$ è una \textbf{stringa} e non un simbolo dell'alfabeto $\Sigma$, quindi $\lambda \not\in \Sigma$.
    \item $\Sigma^*$ è l'insieme di tutte le possibili stringhe sull'alfabeto $\Sigma$.
    \item L'insieme $\Sigma^*$ è infinito e $\lambda \in \Sigma^*$.
\end{itemize}

\section{Definizione ricorsiva di $\Sigma^*$}
\textbf{Passo base}: $\lambda \in \Sigma^*$

\textbf{Passo ricorsivo}: Se $w \in \Sigma^*$ e $x \in \Sigma$, allora $wx \in \Sigma^*$
\begin{example}
$\Sigma = \{0, 1\}$ \\
Dal \textbf{Passo base} si ha che $\lambda \in \Sigma^*$. Quindi al \textbf{Passo base} $\Sigma^*=\{\lambda\}$ \\
Applico il passo ricorsivo \\
$w = \lambda \in \Sigma^*, x = 0 \in \Sigma$, allora $wx = \lambda 0 = 0 \in \Sigma^*$. Quindi $\Sigma^*=\{\lambda, 0\}$ \\
Applico il passo ricorsivo \\
$w = \lambda \in \Sigma^*, x = 1 \in \Sigma$, allora $wx = \lambda 1 = 1 \in \Sigma^*$. Quindi $\Sigma^*=\{\lambda, 0, 1\}$ \\
Applico il passo ricorsivo \\
$w = 0 \in \Sigma^*, x = 0 \in \Sigma$, allora $wx = 00 \in \Sigma^*$. Quindi $\Sigma^*=\{\lambda, 0, 1, 00\}$ \\
Applico il passo ricorsivo \\
$w = 0 \in \Sigma^*, x = 1 \in \Sigma$, allora $wx = 01 \in \Sigma^*$. Quindi $\Sigma^*=\{\lambda, 0, 1, 00, 01\}$ \\
...
\end{example}

\section{Lunghezza di una stringa}
\textbf{Passo base}: $|\lambda| = 0$

\textbf{Passo ricorsivo}: Se $w \in \Sigma^*$ e $x \in \Sigma$, allora $|wx| = |w| + 1$

\begin{example}
$|abb| = |ab| + |b| = |a| + |b| + |b| = 1 + 1 + 1 = 3$
\end{example}

\section{Concatenazione di una stringa}
$u$ e $v$ due stringhe, la concatenazione di $u$ e $v$ è la stringa $u \cdot v$. Si indica anche semplicemente con $uv$ senza usare il $\cdot$ \\
\textbf{N.B}: $uv \neq vu$ \\
\textbf{Passo base}: Se $w \in \Sigma^*$, allora $w \lambda = w$ \\
\textbf{Passo ricorsivo}: Se $w_1, w_2 \in \Sigma^*$ e $x \in \Sigma$, allora $w_1 \cdot (w_2x) = (w_1 \cdot w_2)x \in \Sigma^*$
\begin{example}
$abb \cdot ab = abb \cdot (ab) = abba \cdot b = abbab$
\end{example}

\section{Potenza di una stringa}
\textbf{Passo base}: $w^0 = \lambda$

\textbf{Passo ricorsivo}: $w^{n + 1} = w^n \cdot w, \forall n \geq 0$
\begin{example}
$\{(aa)^i \mid 0 \leq i \leq 3\} = \{ \lambda, aa, aaaa, aaaaaa\}$
\end{example}
\begin{example}
$\{aa^i \mid 0 \leq i \leq 3\} = \{ \lambda, aa, aaa, aaaa\}$
\end{example}
\textbf{N.B}: $(aa)^i \neq aa^i$, $(aa)^2 = aaaa, aa^2 = aaa$

\section{Stringa palindroma}
\textbf{Passo base}: $\forall x \in \Sigma$ e $\lambda$ sono stringhe palindrome

\textbf{Passo ricorsivo}: Se $w$ è una stringa palindroma e $x \in \Sigma$, allora $xwx$ è una  stringa palindroma.
\begin{example}
$abba$ \\
Per il \textbf{Passo base} $a, b, \lambda$ sono stringhe palindrome \\
Applico il passo ricorsivo \\
$w = \lambda$ palindroma per il \textbf{Passo base} e $x = b \in \Sigma$, allora $xwx = b \lambda b = bb$ è una stringa palindroma. \\
Applico il passo ricorsivo \\
$w = bb$ palindroma per il passo ricorsivo precedente e $x = a \in \Sigma$, allora $xwx = abba$ è una stringa palindroma.
\end{example}

\section{Inversione di una stringa}
\textbf{Passo base}: $\lambda^R = \lambda$

\textbf{Passo ricorsivo}: Se $w \in \Sigma^*$ e $x \in \Sigma$, allora $(wx)^R = xw^R$

\begin{example}
$(abb)^R = b(ab)^R = bb(a)^R = bba$
\end{example}
\begin{example}
$a^R = (\lambda a)^R = a \lambda^R = a \lambda = a$
\end{example}

\section{Ulteriori definizioni di specifiche stringhe}
\subsection{Stringhe su $\{a, b\}$ di lunghezza pari}
\textbf{Passo base}: $\lambda \in S$

\textbf{Passo ricorsivo}: Se $w \in S$, allora $waa, wab, wba, wbb \in S$

\subsection{Stringhe pari su $\{a, b\}$ che iniziano con a}
\textbf{Passo base}: $aa, ab \in S$

\textbf{Passo ricorsivo}: Se $w \in S$, allora $waa, wab, wba, wbb \in S$

\subsection{Morfismo}
Prende in input $w \in \{a, b\}^* \setminus \lambda$ e sostituisce $a$ con $0$ e $b$ con $1$

\textbf{Passo base}: $change(a)=0$, $change(b)=1$

\textbf{Passo ricorsivo}: $change(wx)= \begin{cases} change(w)0, \text{ se } x=a \\ change(w)1, \text{ se } x=b \end{cases}$