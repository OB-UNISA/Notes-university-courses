\section{Perché studiare la logica proposizionale}
La materia insegna a distinguere quelle che sono delle proposizioni, ossia che hanno valore \textbf{True} \underline{\textit{oppure}} \textbf{False}, e quelle che invece non lo sono.

La frase \emph{Che bella giornata} non è una proposizione, in quanto essa può essere sia True che False.

Allo stesso modo affermazioni matematiche del tipo $x + 2 = 5$ non è una proposizione in quanto il valore della \textit{x} può variare e con esso varia il valore di verità della proposizione in True o False.

Invece proposizioni del tipo \emph{La penna è sul tavolo} oppure $5 + 4 = 0$ sono proposizioni in quanto hanno un valore di verità univoco al momento della loro affermazione.

\section{Proposizioni}
\subsection{Semplici}
Le proposizioni semplici sono dette tali quando hanno un valore \textbf{T} o \textbf{F} e non sono usati connettivi logici.
\begin{example}
\emph{La televisione è accesa}
\end{example}
\begin{example}
$3 + 5 = 8$
\end{example}

\subsection{Composte}
Le proposizioni composte sono dette tali quando hanno un valore \textbf{T} o \textbf{F} e sono usati connettivi logici.
\begin{example}
\emph{Laura fa i compiti \textbf{e} ascolta la musica}
\end{example}
\begin{example}
\emph{Se domani piove \textbf{allora} prenderò l'ombrello}
\end{example}

\section{Connettivi logici}

\subsection{NOT}
Il connettivo logico $\neg$ ha valore \textbf{T} se e solo se la proposizione \textit{p} ha valore \textbf{F}, e ha valore \textbf{F} se e solo se la proposizione \textit{p} ha valore \textbf{T}.

\begin{table}[H]
    \centering
    \caption{\label{tab:true_table_NOT}Tabella di verità del connettivo logico NOT.}
    \begin{tabular}{|c || c ||} 
     \hline
     \textit{p} & $\neg p$ \\
     \hline\hline
     T & F \\ 
     \hline
     F & T \\
     \hline
    \end{tabular}
\end{table}

\subsection{AND}
Il connettivo logico \textbf{$\wedge$} ha valore \textbf{T} se e solo se entrambe le proposizioni \textit{p} e \textit{q} hanno valore \textbf{T}, se una delle due è \textbf{F} allora il valore della proposizione è \textbf{F}.

\begin{table}[H]
    \centering
    \caption{\label{tab:true_table_AND}Tabella di verità del connettivo logico AND.}
    \begin{tabular}{|c | c || c ||} 
     \hline
     \textit{p} & \textit{q} & $p \wedge q$ \\
     \hline\hline
     T & T & T \\ 
     \hline
     T & F & F \\
     \hline
     F & T & F \\
     \hline
     F & F & F \\
     \hline
    \end{tabular}
\end{table}

\subsection{OR}
Il connettivo logico \textbf{$\vee$} ha valore \textbf{T} se e solo se una delle proposizioni \textit{p} o \textit{q} ha valore \textbf{T}, se sono entrambe \textbf{F} allora il valore della proposizione è \textbf{F}.

\begin{table}[H]
    \centering
    \caption{\label{tab:true_table_OR}Tabella di verità del connettivo logico OR.}
    \begin{tabular}{|c | c || c ||} 
     \hline
     \textit{p} & \textit{q} & $p \vee q$ \\
     \hline\hline
     T & T & T \\ 
     \hline
     T & F & T \\
     \hline
     F & T & T \\
     \hline
     F & F & F \\
     \hline
    \end{tabular}
\end{table}

\subsection{XOR}
Il connettivo logico \textbf{$\oplus$} ha valore \textbf{T} se e solo se una delle le proposizioni \textit{p} o \textit{q} ha valore \textbf{T} ma non entrambe, se sono entrambe \textbf{F} o \textbf{T} allora il valore della proposizione è \textbf{F}.

\begin{table}[H]
    \centering
    \caption{\label{tab:XOR}Tabella di verità del connettivo logico XOR.}
    \begin{tabular}{|c | c || c ||} 
     \hline
     \textit{p} & \textit{q} & $p \oplus q$ \\
     \hline\hline
     T & T & F \\ 
     \hline
     T & F & T \\
     \hline
     F & T & T \\
     \hline
     F & F & F \\
     \hline
    \end{tabular}
\end{table}

\subsection{Implicazione}
Il connettivo logico \textbf{$\implies$} ha valore \textbf{F} se e solo se la proposizione \textit{p} ha valore \textbf{T} e la proposizione \textit{q} ha valore \textbf{F}, altrimenti il valore della proposizione è \textbf{T}.

\begin{table}[H]
    \centering    
    \caption{\label{tab:true_table_implies}Tabella di verità del connettivo logico Implicazione.}
    \begin{tabular}{|c | c || c ||} 
     \hline
     \textit{p} & \textit{q} & $p \implies q$ \\
     \hline\hline
     T & T & T \\ 
     \hline
     T & F & F \\
     \hline
     F & T & T \\
     \hline
     F & F & T \\
     \hline
    \end{tabular}
\end{table}

\subsection{Equivalenza}
Il connettivo logico \textbf{$\iff$} ha valore \textbf{T} se e solo se la proposizione \textit{p} e la proposizione \textit{q} hanno valori uguali, altrimenti il valore della proposizione è \textbf{F}.

\begin{table}[H]
    \centering
    \caption{\label{tab:true_table_iff}Tabella di verità del connettivo logico Equivalenza.}
    \begin{tabular}{|c | c || c ||} 
     \hline
     \textit{p} & \textit{q} & $p \iff q$ \\
     \hline\hline
     T & T & T \\ 
     \hline
     T & F & F \\
     \hline
     F & T & F \\
     \hline
     F & F & T \\
     \hline
    \end{tabular}
\end{table}

\section{Equivalenza logica}
Due proposizioni composte \textit{p} e \textit{q} si dicono equivalenti logicamente se e solo se hanno la stessa tabella di verità e si indica con \textbf{p $\equiv$ q}.\\
\textbf{N.B:} $\equiv$ non è un connettivo logico. \\
\textbf{N.B:} $\equiv$ e $\iff$ sono diversi. \\
\textbf{N.B:} Non si usa il simbolo $=$ per esprimere l'equivalenza logica, ma si usa il simbolo $\equiv$.

\subsection{Tautologia}
La tautologia è una proposizione composta sempre \textbf{True}, ossia tutte \textbf{T} nella colonna della proposizione composta, qualsiasi sia il valore delle proposizioni elementari che la compongono. \\
\begin{table}[H]
    \centering    
    \caption{\label{tab:true_table_tautologia}Tabella di verità della tautologia $(p \wedge q) \implies (p \vee q)$.}
    \begin{tabular}{|c | c | c | c || c ||} 
     \hline
     \textit{p} & \textit{q} & $p \wedge q$ & $p \vee q$ & $(p \wedge q) \implies (p \vee q)$ \\
     \hline\hline
     T & T & T & T & T \\ 
     \hline
     T & F & F & T & T \\
     \hline
     F & T & F & T & T \\
     \hline
     F & F & F & F & T \\
     \hline
    \end{tabular}
\end{table}

\subsection{Contraddizione}
La contraddizione è una proposizione composta sempre \textbf{False}, ossia tutte \textbf{F} nella colonna della proposizione composta, qualsiasi sia il valore delle proposizioni elementari che la compongono.
\begin{table}[H]
    \caption{\label{tab:true_table_contraddizione}Tabella di verità della contraddizione $p \wedge \neg p$.}
    \centering
    \begin{tabular}{|c | c || c ||} 
     \hline
     \textit{p} & $\neg p$ & $p \wedge \neg p$ \\
     \hline\hline
     T & F & F \\ 
     \hline
     F & T & F \\
     \hline
    \end{tabular}
\end{table}

\section{Inverso, opposto e contronominale}
\subsection{Inverso}
Inverso di $p \implies q$ è $q \implies p$ \\
$p \implies q \not\equiv q \implies p$
\begin{table}[H]
    \centering    
    \caption{\label{tab:true_table_inverso}Tabella di verità di $p \implies q \not\equiv q \implies p$.}
    \begin{tabular}{|c | c | c || c ||} 
     \hline
     \textit{p} & \textit{q} & $p \implies q$ & $q \implies p$ \\
     \hline\hline
     T & T & T & T\\ 
     \hline
     T & F & F & T\\
     \hline
     F & T & T  & F\\
     \hline
     F & F & T & T \\
     \hline
    \end{tabular}
\end{table}
\begin{example}
\emph{Se domani piove \textbf{allora} prenderò l'ombrello} \\
Inverso: \emph{Prenderò l'ombrello \textbf{se} domani piove}
\end{example}

\subsection{Opposto}
Opposto di $p \implies q$ è  $\neg p \implies \neg q$ \\
$p \implies q \not\equiv \neg p \implies \neg q$
\begin{table}[H]
    \centering    
    \caption{\label{tab:true_table_opposto}Tabella di verità di $p \implies q \not\equiv \neg p \implies \neg q$.}
    \begin{tabular}{|c | c | c | c | c || c ||} 
     \hline
     \textit{p} & \textit{q} & $\neg p$ & $\neg q$ & $p \implies q$ & $\neg p \implies \neg q$ \\
     \hline\hline
     T & T & F & F & T & T\\ 
     \hline
     T & F & F & T & F & T\\
     \hline
     F & T & T & F & T & F\\
     \hline
     F & F & T & T & T & T\\
     \hline
    \end{tabular}
\end{table}
\begin{example}
\emph{Se domani piove \textbf{allora} prenderò l'ombrello} \\
Opposto: \emph{Se domani non piove \textbf{allora} non prenderò l'ombrello}
\end{example}

\subsection{Contronominale}
Contronominale di $p \implies q$ è  $\neg q \implies \neg p$ \\
$p \implies q \equiv \neg p \implies \neg q$
\begin{table}[H]
    \centering    
    \caption{\label{tab:true_table_contronominale}Tabella di verità di $p \implies q \equiv \neg p \implies \neg q$.}
    \begin{tabular}{|c | c | c | c | c || c ||} 
     \hline
     \textit{p} & \textit{q} & $\neg p$ & $\neg q$ & $p \implies q$ & $\neg q \implies \neg p$ \\
     \hline\hline
     T & T & F & F & T & T\\ 
     \hline
     T & F & F & T & F & F\\
     \hline
     F & T & T & F & T & T\\
     \hline
     F & F & T & T & T & T\\
     \hline
    \end{tabular}
\end{table}
\begin{example}
\emph{Se domani piove \textbf{allora} prenderò l'ombrello} \\
Contronominale: \emph{Non prenderò l'ombrello \textbf{se} domani non piove}
\end{example}

\subsection{Osservazione: l'inverso è equivalente logicamente all'opposto}
Dalla \autoref{tab:true_table_inverso} e dalla \autoref{tab:true_table_opposto} si ha che $q \implies p \equiv \neg p \implies \neg q$
\begin{table}[H]
    \centering    
    \caption{\label{tab:true_table_inverso_eqlogic_opposto}Tabella di verità di $q \implies p \equiv \neg p \implies \neg q$.}
    \begin{tabular}{|c | c | c | c | c || c ||} 
     \hline
     \textit{p} & \textit{q} & $\neg p$ & $\neg q$ & $q \implies p$ & $\neg p \implies \neg q$ \\
     \hline\hline
     T & T & F & F & T & T\\ 
     \hline
     T & F & F & T & T & T\\
     \hline
     F & T & T & F & F & F\\
     \hline
     F & F & T & T & T & T\\
     \hline
    \end{tabular}
\end{table}

\section{Equivalenze logiche note}
\subsection{Leggi di De Morgan}
\label{subs:de_morgan_laws}
\begin{itemize}
    \item $\neg(p \wedge q) \equiv \neg p \vee \neg q$
    \item $\neg(p \vee q) \equiv \neg p \wedge \neg q$
    \item $p \implies q \equiv \neg p \vee q$
    \item $\neg(p \implies q) \equiv \neg(\neg p \vee q) \equiv p \wedge \neg q$
\end{itemize}

\subsection{Equivalenze più usate}
\begin{itemize}
    \item $\neg(\neg p) \equiv p$
    \item $p \vee (q \wedge r) \equiv (p \vee q) \wedge (p \vee r)$
    \item $p \wedge (q \vee r) \equiv (p \wedge q) \vee (p \wedge r)$
    \item $(x < y < z) \equiv (x < y) \wedge (y < z)$
    \item $p \iff q \equiv (p \implies q) \wedge (q \implies p) \equiv (\neg p \vee q) \wedge (\neg q \vee p)$
\end{itemize}

\subsection{Altre equivalenze}
\begin{itemize}
    \item $p \wedge T \equiv p$
    \item $p \vee F \equiv p$
    \item $p \vee T \equiv T$
    \item $p \wedge F \equiv F$
    \item $p \vee p \equiv p$
    \item $p \wedge p \equiv p$
    \item $p \vee \neg p \equiv T$
    \item $p \wedge \neg p \equiv F$
\end{itemize}
