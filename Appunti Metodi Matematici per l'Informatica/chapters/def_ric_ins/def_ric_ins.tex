\section{Insiemi numerici}
Applicare il passo ricorsivo un paio di volte, capire cosa contiene l'insieme e darne la costruzione ricorsiva. La correttezza della costruzione ricorsiva sarà dimostrata successivamente con il \nameref{cha:principio_induzione}.
\begin{example}
\phantom{}
\centerline{\textbf{Definizione ricorsiva}}
\textbf{Passo base}: $1 \in A$ \\
\textbf{Passo ricorsivo}: Se $x \in A$, allora $x + 2 \in A$ \\ 
\centerline{\textbf{Costruzione ricorsiva}}
Applico il passo ricorsivo \\
$x = 1 \in A$, allora $x + 2 = 1 + 2 = 3 \in A$ \\
Applico il passo ricorsivo \\
$x = 3 \in A$, allora $x + 2 = 3 + 2 = 5 \in A$ \\

Si può notare che \textit{A} contiene i numeri dispari, quindi la sua costruzione ricorsiva è: $A = \{2n + 1 \mid n \geq 0, n \in \mathbb{N}\}$
\end{example}

\begin{example}
\phantom{}
\centerline{\textbf{Definizione ricorsiva}}
\textbf{Passo base}: $3 \in S$ \\
\textbf{Passo ricorsivo}: Se $x, y \in S$, allora $x + y \in S$ \\
\centerline{\textbf{Costruzione ricorsiva}}
Applico il passo ricorsivo \\
$x = 3 \in S, y = 3 \in S$, allora $x + y = 3 + 3 = 6 \in S$ \\
Applico il passo ricorsivo \\
$x = 6 \in S, y = 3 \in S$, allora $x + y = 6 + 3 = 9 \in S$ \\
Applico il passo ricorsivo \\
$x = 9 \in S, y = 3 \in S$, allora $x + y = 9 + 3 = 12 \in S$ \\
Applico il passo ricorsivo \\
$x = 9 \in S, y = 6 \in S$, allora $x + y = 9 + 6 = 15 \in S$ \\

Si può notare che \textit{S} contiene i numeri multipli di 3, quindi la sua costruzione ricorsiva è: $S = \{3n \mid n \geq 1, \in \mathbb{N}\}$
\end{example}

\begin{example}
\phantom{}
\centerline{\textbf{Definizione ricorsiva}}
\textbf{Passo base}: $1 \in T$ \\
\textbf{Passo ricorsivo}: Se $x \in T$, allora $3 \cdot x \in T$ \\
\centerline{\textbf{Costruzione ricorsiva}}
Applico il passo ricorsivo \\
$x = 1 \in T$, allora $3 \cdot x = 3 \cdot 1 = 3 \in T$ \\
Applico il passo ricorsivo \\
$x = 3 \in T$, allora $3 \cdot x = 3 \cdot 3 = 9 \in T$ \\
Applico il passo ricorsivo \\
$x = 9 \in T$, allora $3 \cdot x = 3 \cdot 9 = 27 \in T$ \\

Si può notare che \textit{T} contiene i numeri $3^n$, quindi la sua costruzione ricorsiva è: $T = \{3^n \mid n \in \mathbb{N}\}$
\end{example}