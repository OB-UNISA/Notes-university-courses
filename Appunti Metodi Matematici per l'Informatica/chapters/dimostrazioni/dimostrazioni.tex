\section{Dimostrazione diretta}
$p \implies q$
\begin{example}
\emph{Se \textit{n} è dispari allora $n^2$ è dispari.} \\
\textbf{Dimostrazione}: Sia \textit{n} un intero dispari. \\
$n^2=(2k+1)^2=4k^2+1+4k=2(2k^2+2k)+1$ \\
Quindi $n^2$ è dispari.
\end{example}

\section{Dimostrazione per contrapposizione}
\label{sec:dimostrazione_contrapposizione}

$\neg q \implies \neg p$ \\
Dalla \autoref{tab:true_table_contronominale} si ha che $\neg q \implies \neg p \equiv p \implies q$
\begin{example}
Se $3n+2$ è dispari allora \textit{n} è dispari. \\
\textbf{Contronominale}: Se \textit{n} è pari allora $3n+2$ è pari. \\
\textbf{Dimostrazione}: Sia n pari. \\
$3n+2=3(2k)+2=2(3k)+2=2(3k+1)$ \\
$2(3k+1)$ è sempre un numero pari in quanto $2 \cdot r, r \in \mathbb{R}$ è sempre pari. \\
Avendo dimostrato che $\neg q \implies \neg p$ è \textbf{True} e poiché $\neg q \implies \neg p \equiv p \implies q$, si ha che \emph{$3n+2$ è dispari allora \textit{n} è dispari} è \textbf{True}.
\end{example}

\section{Dimostrazione per assurdo o contraddizione}
$p \implies q$ è \textbf{T} quando $p \wedge \neg q$ è \textbf{F} \\
Si ricorda che dalle \nameref{subs:de_morgan_laws} si ha che $\neg (p \implies q) \equiv p \wedge \neg q$
\begin{table}[H]
    \centering    
    \caption{\label{tab:true_table_p_imp_q__p_and_not_q}Tabella di verità di $p \implies q$ e $p \wedge \neg q$.}
    \begin{tabular}{|c | c | c | c || c ||} 
     \hline
     \textit{p} & \textit{q} & $\neg q$ & $p \implies q$ & $p \wedge \neg q$ \\
     \hline\hline
     T & T & F & T & F \\ 
     \hline
     T & F & T & F & T \\
     \hline
     F & T & F & T & F \\
     \hline
     F & F & T & T & F \\
     \hline
    \end{tabular}
\end{table}
\begin{example}
\emph{Se $3n+2$ è dispari allora \textit{n} è dispari}. \\
\textbf{$p \wedge \neg q$}: $3n+2$ è dispari e \textit{n} è pari. \\
\textbf{Dimostrazione}: \\
$3n+2=3(2k)+2=2(3k+1)$ \\
Nella \nameref{sec:dimostrazione_contrapposizione} abbiamo visto che $2(3k+1)$ è sempre pari. Poiché $3n+2=2(3k+1)$ allora $3n+2$ è pari, ma avevamo detto che era dispari. Quindi è un assurdo e \textbf{$p \wedge \neg q$} è \textbf{F}, quindi $p \implies q$ è \textbf{T}, cioè \emph{Se $3n+2$ è dispari allora \textit{n} è dispari} è \textbf{T}.
\end{example}

\section{Dimostrazione banale o vuota}
$p \implies q$, \textit{p}=\textbf{F} allora dalla \autoref{tab:true_table_implies} $p \implies q$ è sempre \textbf{True}.

\section{Dimostrazione di equivalenza}
$p \iff q$ \\
Si dimostra $(p \implies q) \wedge (q \implies p)$

\section{Controesempio}
$\forall x P(x)$ \\
$\neg(\forall x P(x)) \equiv \exists x\neg P(x)$

\section{Prova di esistenza}
$\exists x P(x)$
\begin{itemize}
    \item Può essere costruttiva esibendo un elemento \textit{x} nel dominio per cui $P(x)$ è \textbf{True}.
    \item Può essere non costruttiva.
\end{itemize}

\section{Dimostrazione per casi}
$(p_1 \vee p_2 \vee ... \vee p_n) \implies q \equiv (p_1 \implies q) \vee (p_2 \implies q) \vee ... \vee (p_n \implies q)$

\section{Dimostrazione esaustiva}
<<TODO>>