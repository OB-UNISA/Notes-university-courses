\section{Funzione matematica in $\mathbb{N}$}
\textbf{Passo base}: Specificare il valore della funzione in $0$.

\textbf{Passo ricorsivo}: Definire la regola per ottenere il valore della funzione su un intero attraverso i valori della funzione su interi più piccoli.

\begin{example}
$f: \mathbb{N} \longrightarrow \mathbb{N}$ \\
\textbf{Passo Base}: $f(0) = 3$ \\
\textbf{Passo ricorsivo}: $f(n + 1) = 2f(n) + 3$
\end{example}

La funzione target, ad esempio $f(3)$, si può calcolare buttom up o top down.

\subsection{Calcolo buttom up}
Si parte dal \textbf{Passo base}, in quanto si conosce il valore della funzione e si arriva a ogni iterazione al valore della funzione target.
\begin{example}
$f(0) = 3 \\
f(1) = 2 \cdot 3 + 3 = 9 \\
f(2) = 2 \cdot 9 + 3 = 21 \\
f(3) = 2 \cdot 21 + 3 = 45$
\end{example}

\subsection{Calcolo top down}
Si parte dal target, ad esempio $f(3)$, e si scrive la sua funzione ricorsiva. Si scende fino al \textbf{Passo base} in quanto non si conosce il valore delle precedenti funzioni ricorsive. Arrivati al \textbf{Passo base}, si inizia a scrivere il valore delle funzioni che si conoscono a ritroso, in questo modo si arriverà al valore della funzione target.
\begin{example}
$f(3) = 2f(2) + 3 = 2 \cdot 21 + 3 = 45 \\
f(2) = 2f(1) + 3 = 2 \cdot 9 + 3 = 21 \\
f(1) = 2f(0) + 3 = 2 \cdot 3 + 3 = 9 \\
f(0) = 3$
\end{example}

\section{Dalla definizione ricorsiva alla funzione ricorsiva}
Per passare dalla definizione ricorsiva alla funzione ricorsiva bisogna scoprire il meccanismo ricorsivo. Lo si può fare calcolando il \textbf{Passo base}, la funzione in \textit{n}, cioè $f(n)$, e la funzione in $f(n + 1)$.

\textbf{N.B}: la funzione ricorsiva è $f(n + 1)$. \\
\textbf{N.B}: nella funzione ricorsiva $f(n + 1)$ ci deve essere sempre una chiamata a valori di funzioni precedenti a $f(n + 1)$, altrimenti non è una funzione ricorsiva.
\begin{example}
Fibonacci \\
\centerline{\textbf{Definizione ricorsiva}} \\
\textbf{Passo base}: $F_0 = F_1 = 1$ \\
\textbf{Passo ricorsivo}: $F_n = F_{n - 1} + F_{n - 2}, n \geq 2$ \\
\centerline{\textbf{Funzione ricorsiva}}
$f(0) = 1 \\
f(1) = 1 \\
f(n) = f(n - 1) + f(n - 2) \\ 
f(n + 1) = f((n + 1) - 1) + f((n + 1) - 2) = f(n) + f(n - 1)$ 
\end{example}

\begin{example}
$f: \mathbb{N} \longrightarrow \mathbb{N}$ \\
\centerline{\textbf{Definizione ricorsiva}} \\
$f(n) = a^n$ \\
\centerline{\textbf{Funzione ricorsiva}}
$f(0) = a^0 = 1 \\
f(n) = a^n \\ 
f(n + 1) = a^{n + 1} = a \cdot a^n = a \cdot f(n)$ \\
\textbf{N.B.}: Se avessimo scritto solamente $f(n + 1) = a^{n + 1} = a \cdot a^n$, non è corretto, in quanto non è una definizione ricorsiva poiché non vi sono chiamate ai valori delle funzioni precedenti a $f(n + 1)$.
\end{example}
